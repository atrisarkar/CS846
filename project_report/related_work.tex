% !TEX encoding = UTF-8 Unicode
% !TEX root = project.tex

\section{Related Work}
\label{sec:related}

There has been some previous research which has proposed the idea of mining repositories to find simultaneously checked-in files to help developers in various ways. Zimmermann, Thomas, et al.~\cite{zimmermann2005mining} and Ying, Annie TT, et al.~\cite{ying2004predicting} have looked at exactly the same idea and proposed to use association rule mining to dig out related files from the repository checked-in data. However, both their strategies were designed keeping CVS in mind as the source control system and they had to do a lot of preprocessing on the data to fit it in their algorithm. Over the last decade, there has been increasing focus on maintaining more metadata information in source control system. For example, Github has an open API which can be utilised to get commit information for any open source project in their system in a structured way. Our first objective is to study and review Zimmerman's and Ying's approach in the light of recent developments in open source repository systems and analyse how much of that research can be translated into today's systems and if we can enhance their methodology using this new information which we have today.

Hatari is similar to our work in a way that it shows the location of files which has a risk of later problems by relating its version history to a bug database and finding out whether there were risks in the past~\cite{sliwerski2005hatari}. Our work, on the other hand, looks through the commit history and look for files from earlier commits that were checked-in along with the file in question to find the related files that might have impact in the future using the Top-K association rule mining algorithm~\cite{fournier2012mining}. Hatari is static and does not adapt to new change information~\cite{kim2007predicting}, whereas our work adapts as more and more files are added to the pull requests.

Walker, Robert J., et al.~\cite{walker2006lightweight} proposed an approach to assess and communicate technical risks, which is based on weakly-estimated knowledge, historical change behavior, and the current structure of the software. The approach is built on a probabilistic algorithm, which is combined with dependency analysis and history mining~\cite{lehnert2011review}. CVS repositories were used for their study. In our work, we make use of Github repositories and the algorithm we use is not a probabilistic one.






