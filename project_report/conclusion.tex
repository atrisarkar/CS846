% !TEX encoding = UTF-8 Unicode
% !TEX root = project.tex

\section{Conclusion}
\label{sec:conclusion}

Maintaining the high quality of a software system is a difficult task, especially for systems where there are several contributors. Open source software projects face the challenging task of upholding a high level of quality given the amount of developers that have the ability to make changes to these systems. We presented the RICE tool (Review Impact with ConfidencE) that helps developers and reviewers gain an understanding of the impact that changes may make in a software system. We integrated the RICE tool with the GitHub Pull Request UI to allow for a seamless experience when reviewing changes. RICE uses file commit history data and Top-K association rule mining in order to gather files of interest that may be related to the files in question. We evaluated the validity of our work by testing our tool against several pull request samples and found that roughly 1 in 5 files that are newly modified after the first commit may be recommended by RICE. Our hope is that this work helps developers and project reviewers in their ability to identify the impact of modified files in order to maintain a high quality system.\\
\subsection{Future Work}
In this paper we mined simultaneous file commits to calculate the impact of files on each other. In future we would like to explore other techniques like static analysis and structural relations to measure the impact and see how they compare to our existing method. Also, we need to find better way to contextualise the impact. Currently our tool shows just the last defect in which the impacted file and the changed file was checked-in together. However, we would like to find better summarization techniques to display the context in a more intuitive fashion.

